% super simple template for automated 2022 ADASS manuscript generation from the registration entry
% place this file in your ADASS2022_author_template directory together with your makedefs file
%
% Only few comments here, see the ADASS_template.tex for a more fully commented version, and
% ManuscriptInstructions.pdf if you need more background, and if you even need more, APS's own
% manual2010.pdf has it all!

% Version 3-oct-2022 (Severin Gaudet and Jeff Burke)

\documentclass[11pt,twoside]{article}
\usepackage{asp2014}

\aspSuppressVolSlug
\resetcounters

\bibliographystyle{asp2014}

\markboth{Tocknell and Lorente}{Ensuring continuing trust in our numerical ecosystem}
% remove/add as you need

\begin{document}

\title{Ensuring continuing trust in our numerical ecosystem}

% full name: James Tocknell
\author{James~Tocknell$^1$ and Nuria~Lorente$^1$}
\affil{$^1$Australian Astronomical Optics - Macquarie, Faculty of Science and
Engineering, Macquarie University, North Ryde, NSW 2113, Australia; \email{james.tocknell@mq.edu.au}}
% remove/add as you need

% remove/add authors as you need
\paperauthor{James~Tocknell}{james.tocknell@mq.edu.au}{0000-0001-6637-6922}{}{Australian
  Astronomical Optics - Macquarie, Faculty of Science and Engineering, Macquarie
  University}{North Ryde}{NSW}{2113}{Australia}
\paperauthor{Nuria~Lorente}{nuria.lorente@mq.edu.au}{0000-0003-0450-4807}{}{Australian
  Astronomical Optics - Macquarie, Faculty of Science and Engineering, Macquarie
  University}{North Ryde}{NSW}{2113}{Australia}
% remove/add as you need

% leave these next few aindex lines commented for the editors to enable them. Use Aindex.py to generate them for yourself.
% first presenting author should be the first entry for bold-facing the author index page-reference
%\aindex{Tocknell,~J.}
%\aindex{Author2,~S.}
% remove/add as you need

% leave the ssindex lines commented for the editors to enable them, use Index.py to suggest yours
%\ssindex{FOOBAR!conference!ADASS 2022}
%\ssindex{FOOBAR!organisations!ASP}

% leave the ooindex lines commented for the editors to enable them, use ascl.py to suggest yours
%\ooindex{FOOBAR, ascl:1101.010}
  
\begin{abstract}

Many of the tools we develop build depend on core numerical libraries such as
FFTW, or more generally ecosystems such as those from netlib or scipy. Whilst in
many cases these core libraries are well regarded and used, the need to either
run tools in new environments (such as client-side in a web browser or on a
mobile device), or with the rise of new challengers to the current Fortran/C/C++
ecosystem, such as Rust, Julia or possibly even Go, means that either these
libraries are being ported, or new libraries being written wholesale. This BoF
aims to start the conversation around what we can do to ensure these new
libraries are trustworthy, by firstly covering some of the experience of the BoF
organisers, and then opening up a wider discussion.

\end{abstract}

\section{Motivation}
The catalyst for this Birds of a Feather session was one of the authors needing
to integrate the automatic redshift determination of MARZ
\citep{2016ascl.soft05001H} into an existing Python web application. In an
attempt to keep the Python web application maintainable, the MARZ redshifting
algorithm was ported to Python, with the existing tests that came with MARZ used
to validate the code. Differences in the output of the different implementations
soon became apparent, which were isolated to two of MARZ's dependencies:
\texttt{DSP.js} and \texttt{Regression.js}. These libraries produced incorrect
output,\footnote{We plan to present a detailed look at the process of discovery
at a future ADASS, but the plot in fig.~\ref{B07_f1.eps} provides a taste of the
issues with these libraries.} and so the issue was raised of how do we, as the
Research Software Engineers of the astronomy community, ensure that our software
ecosystem remains trustworthy.

\articlefigure{B07_f1.eps}{B07_f1}{A plot of the different stages of the MARZ
  code to a flat line with two peaks (one of the provided test cases). Note the
  introduction of artefacts and their further propagation through the pipeline.}

\section{Issues seen by participants}
The BoF started with James Tocknell providing a overview of the issues with
MARZ, but also the example of numerically solving a Cubic equation: using the
algorithm given in \citet{1992nrfa.book.....P} gave fairly inaccurate results,
whereas using an algorithm from William Kahan (of IEEE 754 fame) gives accurate
results down to numerical precision.\footnote{The algorithm used is detailed in
  \url{https://people.eecs.berkeley.edu/~wkahan/Math128/Cubic.pdf}, which seems
  to be from a course that was taught.}.

Another case study given was the use of the various versions of the code knows
as \texttt{mpfit}, as it has been converted and upgraded for various programming
languages, with various undocumented modifications along the way. The absence of
a clear provenance, along with the tendency of the code to be
\emph{vendored}\footnote{the process of copying wholesale a codebase into
  another codebase, sometimes with additional modifications} by its users,
implies even if the use of \texttt{mpfit} is documented, the code may not be
behaving correctly.

From this, four key issues were presented:
\begin{itemize}
  \item Where we find algorithms (e.g. papers, textbooks, wikipedia!)?
  \item Where we find implementations of these algorithms?
  \item How we check these implementations are correct?
  \item How we ensure the community uses these correct implementations, not
    incorrect ones?
\end{itemize}

Participants then raised some addition issues:
\begin{itemize}
  \item How can we trust and/or fix closed source codebases such as MATLAB?
  \item How can we handle issues in hardware, as hardware is not immune to these
    issues?
  \item Can a result be considered reproducible if the software used
    to derive it is not available for inspection, and should a distinction be
    made between publishing software for reuse verses publishing software for
    reproducibility?
\end{itemize}

\section{Possible solutions to those issues}
From this starting point, numerous solutions and variations on them were
proposed. A reasonable way of classifying these solutions is by who is
performing them (e.g. code developers, paper authors, journal editors) and to
which codebases the solutions are applied to (e.g. a singular codebase, unrelated
codebases in a single field, unrelated codebases used by an organisation).

One idea that came up in various forms was comparison between codebases solving
a similar problem. These included:
\begin{itemize}
  \item papers where codebases produced results based on a set of benchmarks, with fluid dynamic shocks and dust radiative transfer given as examples of benchmarks
  \item comparisons against a reference or canonical code, with SPICE
    \citep{1996P&SS...44...65A,2018P&SS..150....9A} given as an
    example of where this takes place.
  \item the use of data challenges or competitions, e.g. those run by Kaggle—it
    was suggested one of these could happen at a future ADASS.
  \item the use of a development codebase for experiments and a separate production
    codebase for performance
  \item the use of a different package's test suite
  \item the development of a separate (simpler) codebase which validates more
    complex ones
\end{itemize}

Another set of ideas were around the publication and documentation process.
These included:
\begin{itemize}
  \item citation of the specific algorithm used, and if derived from a different
    codebase, citation of that codebase as well. Scipy
    \citep{2020NatMe..17..261V} was given as an example of a package which does
    this well.
  \item the use of tags for algorithms to aid discovery
  \item the requirement to publish code associated with the paper
  \item the sharing of results of testing codebases by large organisations e.g.
    STScI
  \item detailed information around any limitations of the code, and what the
    target use-cases of the codebase are.  Approximations or speedups which may
    affect precision could be acceptable in parts designed for
    visualisation/sonification/etc. but not for data reduction or analysis.
\end{itemize}

Two community resources were repeatedly brought up: the Astrophysics Source Code
Library (ASCL; \citet{2020ASPC..522..731A}) and the Journal of Open Source
Software (JOSS; ISSN 2475-9066). ASCL is catalogue of codebases used in the
astronomical literature, providing a citable reference where no other exists.
There is no measurement or judgement of quality, nor is the codebase run or
tested, rather it aims to avoid stereotype threat by acknowledging the variable
quality of code produced in research by indexing any codebase used in the
astronomical literature. JOSS complements ASCL by being a journal whose purpose
is to review the quality of the codebase associated with a summary paper, and
thereby improve the quality of scientific codebases.\footnote{One of the authors
  has had two papers published through JOSS, and found it to be a valuable
  experience, and would encourage others to submit their codes to JOSS}

Testing was brought up numerous times (and further discussion of testing
continued in the Testing BoF \citep{B___adassxxxii}). Ideas around testing not already
mentioned included:
\begin{itemize}
  \item the development of a database of tests for specific algorithms
  \item the use of more comprehensive regression test suites
  \item the wider use of property-based testing and fuzzing
\end{itemize}

One final solution that was raised was restrictions around the codebase (either
via licensing or via limiting access to the code to approved people) to prevent
misuse (as has been done by codebases such as pPXF \citep{2017MNRAS.466..798C}).
This approach was views poorly by participants given the unnecessary barriers
created to compare such codes with other codes.

\section{Where next}
The authors expect the discussions around testing to continue to the next ADASS
and beyond, and that the further publication of codebases via the ASCL and JOSS
to be of great benefit to the astronomical community. Adding some kind of
competition or hackathon to ADASS sounds like an excellent idea, but will need
effort from both the hosts (in order to have a viable space for it to take place
in), but also volunteers to run the event.


\acknowledgments The authors would like to thank everyone who participated in
the session, both on zoom via video/audio, and on the session chat.


\bibliography{B07}


% if we have space left, we might add a conference photograph here. Leave commented for now.
% \bookpartphoto[width=1.0\textwidth]{foobar.eps}{FooBar Photo (Photo: Any Photographer)}


\end{document}

